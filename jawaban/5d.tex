Dari problem 5.a didapat persamaan gelombang untuk $\vE$
\begin{align*}
    \nabla^2 \vE&=\mu_0\epsilon_0 \frac{\partial^2 \vE}{\partial t^2}
\end{align*}
Solusi lengkap persamaan gelombang diatas, yakni
\begin{align*}
    \vE(\vr, t)=\vE_0 e^{i(\vk\cdot\vr-\omega t)}
\end{align*}
Subtitusi solusi ini ke dalam persamaan Maxwell ke-3 menghasilkan 
\begin{align*}
    \Curl \vE = -\frac{\partial B}{\partial t} \Rightarrow \vk \times \vE_0 = \omega\vB_0
\end{align*}
yang menyiratkan hubungan antara amplitudo medan listrik dan magnet. Jika vektor gelombang $k$ pada arah $-\hat{z}$, dapat ditulis kembali menjadi
\begin{align*}
    \vB_0=\frac{k}{\omega}(-\hat{z}\times \vE_0)
\end{align*}

% Electromagnetic wave - colored
\begin{center}
    \begin{tikzpicture}[x=(-15:0.9), y=(90:0.9), z=(-150:1.1),
        line cap=round, line join=round,
        axis/.style={black, thick,->},
        axis2/.style={black},
        axis3/.style={black, thick, <-},
        vector/.style={>=stealth,->}]
        \large
        \def\A{1.5}
        \def\nNodes{5} % use even number
        \def\nVectorsPerNode{8}
        \def\N{\nNodes*40}
        \def\xmax{\nNodes*pi/2*1.01}
        \pgfmathsetmacro\nVectors{(\nVectorsPerNode+1)*\nNodes}
        \def\vE{{\color{Ecol}\mathbf{E}}}
        \def\vB{{\color{Bcol}\mathbf{B}}}

        \def\drawENode{ % draw E node and vectors with some offset
        \draw[Ecol,very thick,variable=\t,domain=\iOffset*pi/2:(\iOffset+1)*pi/2*1.01,samples=40]
        plot (\t,{\A*sin(\t*360/pi)},0);
        \foreach \k [evaluate={\t=\k*pi/2/(\nVectorsPerNode+1);
                \angle=\k*90/(\nVectorsPerNode+1);}]
        in {1,...,\nVectorsPerNode}{
        \draw[vector,EVcol]  (\iOffset*pi/2+\t,0,0) -- ++(0,{\A*sin(2*\angle+\iOffset*180)},0);
        }
        }
        \def\drawBNode{ % draw B node and vectors with some offset
        \draw[Bcol,very thick,variable=\t,domain=\iOffset*pi/2:(\iOffset+1)*pi/2*1.01,samples=40]
        plot (\t,0,{\A*sin(\t*360/pi)});
        \foreach \k [evaluate={\t=\k*pi/2/(\nVectorsPerNode+1);
                \angle=\k*90/(\nVectorsPerNode+1);}]
        in {1,...,\nVectorsPerNode}{
        \draw[vector,Bcol!50]  (\iOffset*pi/2+\t,0,0) -- ++(0,0,{\A*sin(2*\angle+\iOffset*180)});
        }
        }

        % MAIN AXES
        \draw[axis] (0,0,0) -- ++(\xmax*1.1,0,0) node[right] {$-z$};
        \draw[axis2, dashed] (0,1.5,0) -- ++(\xmax*1.1,0,0) node[right] {$ $};
        \draw[axis2, dashed] (0,0,1.5) -- ++(\xmax*1.1,0,0) node[right] {$ $};
        \draw[axis] (0,-\A*1.2,0) -- (0,\A*1.6,0) node[right] {$x$};
        \draw[axis] (0,0,-\A*1.2) -- (0,0,\A*1.6) node[left] {$-y$};


        % draw (anti-)nodes
        \foreach \iNode [evaluate={\iOffset=\iNode-1;}] in {1,...,\nNodes}{
        \ifodd\iNode \drawBNode \drawENode % E overlaps B
        \else        \drawENode \drawBNode % B overlaps E
        \fi
        }
        \node[black] at (0.9,2.2,0) {\small $\vec{E}$};
        \node[black] at (0.9,0,2.4) {\small $\vec{B}$};
        \node[black] at (-0.4,1.5,0) {\small $E_0$};
        \node[black] at (-0.6,0,1.5) {\small $E_0/c$};
        \draw[axis] (5,1.8,0) -- ++(1.1,0,0) node[above] {$c$};

    \end{tikzpicture}
\end{center}

$\vE$ dan $\vB$ sefasa dan saling tegak lurus; amplitudo (real) keduanya berhuungan dalam persamaan:
\begin{align*}
    B_0=\frac{k}{\omega}E_0=\frac{1}{c}E_0
\end{align*}

Maka, intensitas medan magnet berdasarkan persamaan yang telah diketahui ($\vE =100 \cos{(\omega t +8 z)\hat{i}}$) adalah
\begin{align*}
    \vB(z,t)&=B_0 \cos{(\omega t +8 z)}\ -\hat{j}\\
    &=\frac{1}{c}E_0 \cos{(\omega t +8 z)}\ -\hat{j}\\
    &=\frac{1}{3\times 10^8}100 \cos{(\omega t +8 z)}\ -\hat{j}\\
    \vB(z,t)&=3.33 \times 10^{-7} \cos{(\omega t +8 z)}\ -\hat{j}
\end{align*}