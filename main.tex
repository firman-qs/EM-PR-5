%\documentclass[12pt]{article}
\documentclass[12pt]{scrartcl}
\title{EM-Pertemuan 7}
\nonstopmode
%\usepackage[utf-8]{inputenc}
\usepackage{graphicx} % Required for including pictures \usepackage[figurename=Figure]{caption} \usepackage{float}    % For tables and other floats
\usepackage{verbatim} % For comments and other
\usepackage{amsmath}  % For math
\usepackage{amssymb}  % For more math
\usepackage{amsfonts} % For more math fonts
\usepackage{fullpage} % Set margins and place page numbers at bottom center
\usepackage{paralist} % paragraph spacing
\usepackage{listings} % For source code
\usepackage{subfig}   % For subfigures
%\usepackage{physics}  % for simplified dv, and 
\usepackage{enumitem} % useful for itemization
\usepackage{siunitx}  % standardization of si units
\usepackage{esint}
\usepackage{tikz,bm} % Useful for drawing plots
%\usepackage{tikz-3dplot}
\usepackage{cancel}
%%% Colours used in field vectors and propagation direction
\definecolor{mycolor}{rgb}{1,0.2,0.3}
\definecolor{brightgreen}{rgb}{0.4, 1.0, 0.0}
\definecolor{britishracinggreen}{rgb}{0.0, 0.26, 0.15}
\definecolor{cadmiumgreen}{rgb}{0.0, 0.42, 0.24}
\definecolor{ceruleanblue}{rgb}{0.16, 0.32, 0.75}
\definecolor{darkelectricblue}{rgb}{0.33, 0.41, 0.47}
\definecolor{darkpowderblue}{rgb}{0.0, 0.2, 0.6}
\definecolor{darktangerine}{rgb}{1.0, 0.66, 0.07}
\definecolor{emerald}{rgb}{0.31, 0.78, 0.47}
\definecolor{white}{rgb}{255, 255, 255}
\definecolor{palatinatepurple}{rgb}{0.41, 0.16, 0.38}
\definecolor{pastelviolet}{rgb}{0.8, 0.6, 0.79}


% special config
\usepackage{cleveref}
\usepackage[many]{tcolorbox}
\tcbset{
    boxrule=0pt,
    sharp corners,
    enhanced jigsaw,
    % drop fuzzy shadow,
    breakable
}
% some auto vector etc.
\renewcommand{\epsilon}{\varepsilon}
\newcommand{\vnabla}{\vec{\nabla}}
\newcommand{\vE}{\vec{E}}
\newcommand{\vH}{\vec{H}}
\newcommand{\vB}{\vec{B}}
\newcommand{\vD}{\vec{D}}
\newcommand{\vJ}{\vec{J}}

\newtcolorbox{answer}{
    beforeafter skip balanced=5pt,
    colback={white!95!ceruleanblue},
    borderline west={3pt}{0pt}{black!70!white}
}
    

\begin{document}
\renewcommand{\figurename}{Figure}
\begin{center}
    \hrule
    \vspace{.4cm}
    \textbf{\large Elektromagnetika --- PR V}
\end{center}
\textbf{Nama}\hspace{1mm}: Firman Qashdus Sabil \hspace\fill \textbf{Offering:}\ AC\\
 \textbf{NIM}\hspace{3.3mm}: 210321606892 \hspace{\fill} \\%\textbf{Assignment:} Number 3 \\
\hrule

\begin{enumerate}[leftmargin=*]

    \item Menurunkan persamaan gelombang EM dengan kehadiran sumber, untuk medan $\vec{E}$.
    \begin{answer}
        Dari persamaan maxwell ke-3
\begin{equation*}
    \vnabla \times \vE = -\frac{\partial \vB}{\partial t}
\end{equation*}
Curl-kan kedua sis persamaan maxwell diatas,
\begin{align*}
    \vnabla \times \vnabla \times \vE &= \vnabla \times \left(-\frac{\partial \vB}{\partial t}\right)\\
    \vnabla \times \vnabla \times \vE &=  -\frac{\partial}{\partial t}\left(\vnabla \times \vB \right) 
\end{align*}
Dengan identitas vektor, bahwa $\vnabla \times \vnabla \times \vec{A}=\vnabla(\vnabla\cdot\vec{A})-\nabla^2 \vec{A}$ dan karena $\vB = \mu_0 \vH$ maka
\begin{align*}
    \vnabla(\vnabla\cdot\vec{E})-\nabla^2\vE=-\mu_0 \frac{\partial }{\partial t}\left(\vnabla \times \vH\right) 
\end{align*}
karena $\vnabla \times \vH = \frac{\partial \vD}{\partial t} + \vJ$ dengan $\vD = \epsilon_0 \vE$, maka
\begin{align*}
    \vnabla(\vnabla\cdot\vec{E})-\nabla^2\vE&=-\mu_0 \frac{\partial }{\partial t}\left(\vnabla \times \vH\right)\\
    &=-\mu_0 \frac{\partial }{\partial t}\left(\frac{\partial \vD}{\partial t}+\vJ\right) \\
    &=-\mu_0 \frac{\partial }{\partial t}\left(\epsilon_0\frac{\partial \vE}{\partial t}+\vJ\right) \\
    \vnabla(\vnabla\cdot\underbrace{\vec{E}}_{\dfrac{\vD}{\epsilon_0}})-\nabla^2\vE&=-\mu_0\epsilon \frac{\partial^2 }{\partial t^2}\vE-\mu_0\frac{\partial }{\partial t}\vJ
\end{align*}
Dalam kasus ini $\vnabla \cdot \vD = \rho; \text{dimana } \rho \neq 0$, sehingga
\begin{align*}
    \vnabla(\vnabla\cdot\frac{1}{\epsilon_0}\vD)-\nabla^2\vE&=-\mu_0\epsilon \frac{\partial^2 }{\partial t^2}\vE-\mu_0\frac{\partial }{\partial t}\vJ\\
    \frac{1}{\epsilon_0}\vnabla(\underbrace{\vnabla\cdot\vD}_{\rho})-\nabla^2\vE&=-\mu_0\epsilon \frac{\partial^2 }{\partial t^2}\vE-\mu_0\frac{\partial }{\partial t}\vJ\\
\end{align*}
karena $c=\dfrac{1}{\sqrt{\mu_0\epsilon_0}}$, maka
\begin{align*}
    \frac{1}{\epsilon_0}\vnabla\rho-\nabla^2\vE&=-\underbrace{\mu_0\epsilon}_{\dfrac{1}{c^2}} \frac{\partial^2 }{\partial t^2}\vE-\mu_0\frac{\partial }{\partial t}\vJ
\end{align*}
\textit{rearrange} persamaan diatas sehingga menjadi
\begin{align*}
    \nabla^2\vE-\frac{1}{c^2}\frac{\partial^2 }{\partial t^2}\vE&=\mu_0\frac{\partial}{\partial}\vJ + \frac{1}{\epsilon_0}\nabla\rho
\end{align*}
atau
\begin{align*}
    \left(\nabla^2-\frac{1}{c^2}\frac{\partial^2 }{\partial t^2}\right)\vE&=\mu_0\frac{\partial}{\partial}\vJ + \frac{1}{\epsilon_0}\nabla\rho
\end{align*}
    \end{answer}
    \item Menurunkan persamaan gelombang EM dalam medium pengahantra, untuk medan $\vec{E}$ dan $\vec{H}$.
    \begin{answer}
        \begin{itemize}[leftmargin=*]
    \item Untuk medan $\vE$\\
        Dari persamaan maxwell ke-3
        \begin{equation*}
            \vnabla \times \vE = -\frac{\partial \vB}{\partial t}
        \end{equation*}
        Curl-kan kedua sisi persamaan maxwell diatas,
        \begin{align*}
            \vnabla \times \vnabla \times \vE &= \vnabla \times \left(-\frac{\partial \vB}{\partial t}\right)\\
            \vnabla \times \vnabla \times \vE &=  -\frac{\partial}{\partial t}\left(\vnabla \times \vB \right) 
        \end{align*}
        Dengan identitas vektor, bahwa $\vnabla \times \vnabla \times \vec{A}=\vnabla(\vnabla\cdot\vec{A})-\nabla^2 \vec{A}$ dan karena $\vB = \mu_0 \vH$ maka
        \begin{align*}
            \vnabla(\vnabla\cdot\vec{E})-\nabla^2\vE=-\mu_0 \frac{\partial }{\partial t}\left(\vnabla \times \vH\right) 
        \end{align*}
        karena $\vnabla \times \vH = \frac{\partial \vD}{\partial t} + \vJ$ dengan $\vD = \epsilon_0 \vE$, maka
        \begin{align*}
            \vnabla(\vnabla\cdot\vec{E})-\nabla^2\vE&=-\mu_0 \frac{\partial }{\partial t}\left(\vnabla \times \vH\right)\\
            &=-\mu_0 \frac{\partial }{\partial t}\left(\frac{\partial \vD}{\partial t}+\vJ\right) \\
            &=-\mu_0 \frac{\partial }{\partial t}\left(\epsilon_0\frac{\partial \vE}{\partial t}+\vJ\right) \\
            \vnabla(\vnabla\cdot\vec{E})-\nabla^2\vE&=-\mu_0\epsilon_0 \frac{\partial^2 }{\partial t^2}\vE-\mu \frac{\partial }{\partial t}\vJ
        \end{align*}
        Dari hukum Ohm $\vJ=\sigma \vE \neq 0$, untuk $\vE\neq 0$, sehingga persamaan diatas menjadi
        \begin{align*}
            \vnabla(\vnabla\cdot\underbrace{\vec{E}}_{\dfrac{\vD}{\epsilon_0}})-\nabla^2\vE&=-\mu_0\epsilon \frac{\partial^2 }{\partial t^2}\vE-\mu_0\sigma \frac{\partial }{\partial t}\vE\\
            \frac{1}{\epsilon_0}\vnabla(\underbrace{\vnabla\cdot\vD}_{\rho})-\nabla^2\vE&=-\mu_0\epsilon \frac{\partial^2 }{\partial t^2}\vE-\mu_0\sigma \frac{\partial }{\partial t}\vE
        \end{align*}
        Dalam medium konduktor, resistivitas $\rho=0$ dan konduktivitas $\sigma\neq 0$, maka tersisa
        \begin{align*}
            -\nabla^2\vE&=-\mu_0\epsilon_0 \frac{\partial^2 }{\partial t^2}\vE-\mu_0\sigma \frac{\partial }{\partial t}\vE
        \end{align*}
        \textit{rearrange} persamaan diatas menjadi,
        \begin{align*}
            \nabla^2\vE-\mu_0\epsilon_0 \frac{\partial^2 }{\partial t^2}\vE-\mu_0\sigma \frac{\partial }{\partial t}\vE&=0\\
            \left(\nabla^2-\mu_0\epsilon_0 \frac{\partial^2 }{\partial t^2}-\mu_0\sigma \frac{\partial }{\partial t}\right)\vE&=0
        \end{align*}
    \item Untuk medan $\vH$\\
    Dari persamaan Maxwell ke-4
    \begin{align*}
        \vnabla \times \vH = \vJ+\frac{\partial \vD}{\partial t}
    \end{align*}
    Curl-kan kedua ruas persamaan
    \begin{align*}
        \vnabla \times\vnabla \times \vH = \vnabla \times\left(\vJ+\frac{\partial \vD}{\partial t}\right)
    \end{align*}
    Dengan identitas vektor, bahwa $\vnabla \times \vnabla \times \vec{A}=\vnabla(\vnabla\cdot\vec{A})-\nabla^2 \vec{A}$ dan karena $\vB = \mu_0 \vH$ maka
    \begin{align*}
        \vnabla(\vnabla\cdot\underbrace{\vH}_{\dfrac{1}{\mu_0}\vB})-\nabla^2\vH &= \vnabla \times \vJ+\vnabla \times\frac{\partial \vD}{\partial t}\\
        \frac{1}{\mu_0}\vnabla(\vnabla\cdot\vB)-\nabla^2\vH &= \vnabla \times \vJ+\vnabla \times\frac{\partial \vD}{\partial t}
    \end{align*}
    Berdasarkan persamaan Maxwell ke-2 $\vnabla \cdot \vB=0$, maka tersisa
    \begin{align*}
        -\nabla^2\vH &= \vnabla \times \vJ+\vnabla \times\frac{\partial \vD}{\partial t}
    \end{align*}
    Karena $\vD=\epsilon_0 \vE$, maka
    \begin{align*}
        -\nabla^2\vH &= \vnabla \times \vJ + \frac{\partial}{\partial t}\vnabla \times (\epsilon_0 \vE)\\
        &=\vnabla \times \vJ + \epsilon_0\frac{\partial}{\partial t}\underbrace{\vnabla \times \vE}_{\substack{-\frac{\partial \vB}{\partial t} \\ \text{Pers. Ke-3} \\\text{Maxwell}}}\\
        &=\vnabla \times \vJ + \epsilon_0\frac{\partial}{\partial t}\left(-\frac{\partial\vB}{\partial t}\right)\\
        &=\vnabla \times \vJ - \epsilon_0\frac{\partial^2\vB}{\partial t}
    \end{align*}
    karena $\vB=\mu_0 \vH$, maka
    \begin{align*}
        -\nabla^2\vH&=\vnabla \times \vJ - \epsilon_0\mu_0\frac{\partial^2\vH}{\partial t}
    \end{align*}
    Dari hukum Ohm $\vJ=\sigma \vE \neq 0$, untuk $\vE\neq 0$, sehingga persamaan diatas menjadi
    \begin{align*}
        -\nabla^2\vH&=\vnabla \times (\sigma\vE) - \epsilon_0\mu_0\frac{\partial^2\vH}{\partial t}\\
        &=\sigma \underbrace{\vnabla \times \vE}_{\substack{-\frac{\partial \vB}{\partial t} \\ \text{Pers. Ke-3} \\\text{Maxwell}}} - \epsilon_0\mu_0\frac{\partial^2\vH}{\partial t}\\
        -\nabla^2\vH&=-\sigma \frac{\partial \vB}{\partial t} - \epsilon_0\mu_0\frac{\partial^2\vH}{\partial t}
    \end{align*}
    karena $\vB=\mu_0 \vH$, maka
    \begin{align*}
        -\nabla^2\vH&=-\sigma\mu_0 \frac{\partial \vH}{\partial t} - \epsilon_0\mu_0\frac{\partial^2\vH}{\partial t}
    \end{align*}
    \textit{rearrange} persamaan diatas menjadi
    \begin{align*}
        \nabla^2\vH-\sigma\mu_0 \frac{\partial \vH}{\partial t} - \epsilon_0\mu_0\frac{\partial^2\vH}{\partial t}&=0\\
        \left(\nabla^2-\sigma\mu_0 \frac{\partial}{\partial t} - \epsilon_0\mu_0\frac{\partial^2}{\partial t}\right)\vH&=0\\
    \end{align*}
\end{itemize}
    \end{answer}
    \item Diketahui konduktivitas perak $\sigma = 3\times10^7\ \si{S/m}$ pada frekuensi gelombang mikro. Tentukan \textit{skin depth} pada frekuensi $10^{10} \si{Hz}$.
    \begin{answer}
        \textit{skin depth} didefinisikan sebagai jarak untuk mengurangi amplitudo Gelombang EM dengan faktor $1/e$, yakni:
\begin{equation*}
    \delta \equiv \frac{1}{\kappa};\ \text{dimana } \kappa \equiv \omega \sqrt{\frac{\epsilon\mu}{2}} \left[\sqrt{1+\left(\frac{\sigma}{\epsilon\omega}\right)^2}-1\right]^{1/2}
\end{equation*}
Untuk konduktivitas tinggi $(\sigma \gg \omega \epsilon) \Rightarrow \sigma/\epsilon \gg 1$, sehingga
\begin{align*}
    \delta &= \frac{1}{
        \omega \sqrt{\dfrac{\mu \epsilon}{2}} \left[\dfrac{\sigma}{\epsilon\omega}\right]^{1/2}
    }\\
    &=\frac{1}{\sqrt{\dfrac{\omega^{\cancel{2}} \mu \cancel{\epsilon}\sigma}{2\cancel{\epsilon}\cancel{\omega}}}}\\
    &=\frac{1}{\sqrt{\dfrac{\omega\mu\sigma}{2}}}\\
    &=\sqrt{\frac{2}{\omega\mu\sigma}}\\
    &= \sqrt{\frac{2}{2\pi f \mu \sigma}}
\end{align*}
diketahui nilai $f=10^{10}\si{Hz}$, $\sigma=3\times 10^7 \si{S/m}$ dan permeabilitas material $\mu=\mu_0(1+\chi_m)$, dimana suseptabilitas material perak $\chi_m=-2,4\times 10^{-5}$, atau
\begin{align*}
    \mu&=\mu_0(1+\chi_m)\\
    &=4\pi\times 10^{-7}(1+-2.4\times 10^{-5})\\
    &=1,26 \times 10^{-6}\si{N/A^2}
\end{align*}
subtitusi pada persamaan \textit{skin depth} sebelumnya, didapat
\begin{align*}
    \delta&= \sqrt{\frac{2}{2\pi (10^{10}) (1,26 \times 10^{-6}) (3\times 10^7)}}\\
    &= \sqrt{\frac{2}{2.37\times 10^{12}}}\\
    &=9,18 \times 10^{-7} \si{m}\\
    &=0,918\ \si{\mu m} 
\end{align*}
    \end{answer}
    \item Air laut memiliki konduktivitas $\sigma =3\times10^7\ \si{S/m}$ dan $\mu=\mu_0$. Tentukan nilai frekuensi ketika \textit{skin depth}-nya bernilai satu meter. 
    \begin{answer}
        Untuk bahan dengan konduktifitas tinggi maka \textit{skin depth} nya adalah
\begin{align*}
    \delta &= \sqrt{\frac{2}{\omega \mu \sigma}}\\
           &= \sqrt{\frac{2}{2 \pi f \mu \sigma}}
\end{align*}
untuk mencari nilai frekuensi, maka berdasarkan persamaan diatas
\begin{align*}
    \sqrt{2\pi f \mu \sigma} &= \frac{\sqrt{2}}{\delta}\\
    \sqrt{f} &= \frac{\sqrt{2}}{\delta\sqrt{2\pi\mu \sigma}}\\
    f &= \frac{2}{\delta^2 2\pi\mu \sigma}\\
\end{align*}
diketahui bahwa konduktifitas $\sigma=3\times 10^7$, $\mu=\mu_0$, dan $\delta = 1\ \si{m}$, maka
\begin{align*}
    f&=\frac{2}{1^2 (2\pi) (4\pi\times 10^{-7}) (3\times 10^7)}\\
     &=0,00844 Hz\\
     &=8,44 \times 10^{-3}\ \si{Hz}
\end{align*}


    \end{answer}
    \item Intensitas medan listrik yang berbentuk gelombang bidang dalam vakum dinyatakan dengan persamaan sebagai berikut:
    \begin{equation*}
        \vec{E}=100 \cos (\omega t+8z)\hat{i}\ \si{V/m}
    \end{equation*}
    maka tentukan:
    \begin{enumerate}[leftmargin=*]
        \item Kecepatan jalar gelombang
        \item Frekuensi gelombang EM
        \item Panjang gelombang
        \item Intensitas medan magnet
    \end{enumerate}

    \begin{equation*}
        \iiint\limits_{-\infty}^{\infty} \nabla \times E = \frac{1}{4\pi \varepsilon_0}
    \end{equation*}

    % \item Dari persamaan maxwell ke-3
\begin{equation*}
    \vnabla \times \vE = -\frac{\partial \vB}{\partial t}
\end{equation*}
Curl-kan kedua sis persamaan maxwell diatas,
\begin{align*}
    \vnabla \times \vnabla \times \vE &= \vnabla \times \left(-\frac{\partial \vB}{\partial t}\right)\\
    \vnabla \times \vnabla \times \vE &=  -\frac{\partial}{\partial t}\left(\vnabla \times \vB \right) 
\end{align*}
Dengan identitas vektor, bahwa $\vnabla \times \vnabla \times \vec{A}=\vnabla(\vnabla\cdot\vec{A})-\nabla^2 \vec{A}$ dan karena $\vB = \mu_0 \vH$ maka
\begin{align*}
    \vnabla(\vnabla\cdot\vec{E})-\nabla^2\vE=-\mu_0 \frac{\partial }{\partial t}\left(\vnabla \times \vH\right) 
\end{align*}
karena $\vnabla \times \vH = \frac{\partial \vD}{\partial t} + \vJ$ dengan $\vD = \epsilon_0 \vE$, maka
\begin{align*}
    \vnabla(\vnabla\cdot\vec{E})-\nabla^2\vE&=-\mu_0 \frac{\partial }{\partial t}\left(\vnabla \times \vH\right)\\
    &=-\mu_0 \frac{\partial }{\partial t}\left(\frac{\partial \vD}{\partial t}+\vJ\right) \\
    &=-\mu_0 \frac{\partial }{\partial t}\left(\epsilon_0\frac{\partial \vE}{\partial t}+\vJ\right) \\
    \vnabla(\vnabla\cdot\underbrace{\vec{E}}_{\dfrac{\vD}{\epsilon_0}})-\nabla^2\vE&=-\mu_0\epsilon \frac{\partial^2 }{\partial t^2}\vE-\mu_0\frac{\partial }{\partial t}\vJ
\end{align*}
Dalam kasus ini $\vnabla \cdot \vD = \rho; \text{dimana } \rho \neq 0$, sehingga
\begin{align*}
    \vnabla(\vnabla\cdot\frac{1}{\epsilon_0}\vD)-\nabla^2\vE&=-\mu_0\epsilon \frac{\partial^2 }{\partial t^2}\vE-\mu_0\frac{\partial }{\partial t}\vJ\\
    \frac{1}{\epsilon_0}\vnabla(\underbrace{\vnabla\cdot\vD}_{\rho})-\nabla^2\vE&=-\mu_0\epsilon \frac{\partial^2 }{\partial t^2}\vE-\mu_0\frac{\partial }{\partial t}\vJ\\
\end{align*}
karena $c=\dfrac{1}{\sqrt{\mu_0\epsilon_0}}$, maka
\begin{align*}
    \frac{1}{\epsilon_0}\vnabla\rho-\nabla^2\vE&=-\underbrace{\mu_0\epsilon}_{\dfrac{1}{c^2}} \frac{\partial^2 }{\partial t^2}\vE-\mu_0\frac{\partial }{\partial t}\vJ
\end{align*}
\textit{rearrange} persamaan diatas sehingga menjadi
\begin{align*}
    \nabla^2\vE-\frac{1}{c^2}\frac{\partial^2 }{\partial t^2}\vE&=\mu_0\frac{\partial}{\partial}\vJ + \frac{1}{\epsilon_0}\nabla\rho
\end{align*}
atau
\begin{align*}
    \left(\nabla^2-\frac{1}{c^2}\frac{\partial^2 }{\partial t^2}\right)\vE&=\mu_0\frac{\partial}{\partial}\vJ + \frac{1}{\epsilon_0}\nabla\rho
\end{align*}
    % \item \begin{itemize}[leftmargin=*]
    \item Untuk medan $\vE$\\
        Dari persamaan maxwell ke-3
        \begin{equation*}
            \vnabla \times \vE = -\frac{\partial \vB}{\partial t}
        \end{equation*}
        Curl-kan kedua sisi persamaan maxwell diatas,
        \begin{align*}
            \vnabla \times \vnabla \times \vE &= \vnabla \times \left(-\frac{\partial \vB}{\partial t}\right)\\
            \vnabla \times \vnabla \times \vE &=  -\frac{\partial}{\partial t}\left(\vnabla \times \vB \right) 
        \end{align*}
        Dengan identitas vektor, bahwa $\vnabla \times \vnabla \times \vec{A}=\vnabla(\vnabla\cdot\vec{A})-\nabla^2 \vec{A}$ dan karena $\vB = \mu_0 \vH$ maka
        \begin{align*}
            \vnabla(\vnabla\cdot\vec{E})-\nabla^2\vE=-\mu_0 \frac{\partial }{\partial t}\left(\vnabla \times \vH\right) 
        \end{align*}
        karena $\vnabla \times \vH = \frac{\partial \vD}{\partial t} + \vJ$ dengan $\vD = \epsilon_0 \vE$, maka
        \begin{align*}
            \vnabla(\vnabla\cdot\vec{E})-\nabla^2\vE&=-\mu_0 \frac{\partial }{\partial t}\left(\vnabla \times \vH\right)\\
            &=-\mu_0 \frac{\partial }{\partial t}\left(\frac{\partial \vD}{\partial t}+\vJ\right) \\
            &=-\mu_0 \frac{\partial }{\partial t}\left(\epsilon_0\frac{\partial \vE}{\partial t}+\vJ\right) \\
            \vnabla(\vnabla\cdot\vec{E})-\nabla^2\vE&=-\mu_0\epsilon_0 \frac{\partial^2 }{\partial t^2}\vE-\mu \frac{\partial }{\partial t}\vJ
        \end{align*}
        Dari hukum Ohm $\vJ=\sigma \vE \neq 0$, untuk $\vE\neq 0$, sehingga persamaan diatas menjadi
        \begin{align*}
            \vnabla(\vnabla\cdot\underbrace{\vec{E}}_{\dfrac{\vD}{\epsilon_0}})-\nabla^2\vE&=-\mu_0\epsilon \frac{\partial^2 }{\partial t^2}\vE-\mu_0\sigma \frac{\partial }{\partial t}\vE\\
            \frac{1}{\epsilon_0}\vnabla(\underbrace{\vnabla\cdot\vD}_{\rho})-\nabla^2\vE&=-\mu_0\epsilon \frac{\partial^2 }{\partial t^2}\vE-\mu_0\sigma \frac{\partial }{\partial t}\vE
        \end{align*}
        Dalam medium konduktor, resistivitas $\rho=0$ dan konduktivitas $\sigma\neq 0$, maka tersisa
        \begin{align*}
            -\nabla^2\vE&=-\mu_0\epsilon_0 \frac{\partial^2 }{\partial t^2}\vE-\mu_0\sigma \frac{\partial }{\partial t}\vE
        \end{align*}
        \textit{rearrange} persamaan diatas menjadi,
        \begin{align*}
            \nabla^2\vE-\mu_0\epsilon_0 \frac{\partial^2 }{\partial t^2}\vE-\mu_0\sigma \frac{\partial }{\partial t}\vE&=0\\
            \left(\nabla^2-\mu_0\epsilon_0 \frac{\partial^2 }{\partial t^2}-\mu_0\sigma \frac{\partial }{\partial t}\right)\vE&=0
        \end{align*}
    \item Untuk medan $\vH$\\
    Dari persamaan Maxwell ke-4
    \begin{align*}
        \vnabla \times \vH = \vJ+\frac{\partial \vD}{\partial t}
    \end{align*}
    Curl-kan kedua ruas persamaan
    \begin{align*}
        \vnabla \times\vnabla \times \vH = \vnabla \times\left(\vJ+\frac{\partial \vD}{\partial t}\right)
    \end{align*}
    Dengan identitas vektor, bahwa $\vnabla \times \vnabla \times \vec{A}=\vnabla(\vnabla\cdot\vec{A})-\nabla^2 \vec{A}$ dan karena $\vB = \mu_0 \vH$ maka
    \begin{align*}
        \vnabla(\vnabla\cdot\underbrace{\vH}_{\dfrac{1}{\mu_0}\vB})-\nabla^2\vH &= \vnabla \times \vJ+\vnabla \times\frac{\partial \vD}{\partial t}\\
        \frac{1}{\mu_0}\vnabla(\vnabla\cdot\vB)-\nabla^2\vH &= \vnabla \times \vJ+\vnabla \times\frac{\partial \vD}{\partial t}
    \end{align*}
    Berdasarkan persamaan Maxwell ke-2 $\vnabla \cdot \vB=0$, maka tersisa
    \begin{align*}
        -\nabla^2\vH &= \vnabla \times \vJ+\vnabla \times\frac{\partial \vD}{\partial t}
    \end{align*}
    Karena $\vD=\epsilon_0 \vE$, maka
    \begin{align*}
        -\nabla^2\vH &= \vnabla \times \vJ + \frac{\partial}{\partial t}\vnabla \times (\epsilon_0 \vE)\\
        &=\vnabla \times \vJ + \epsilon_0\frac{\partial}{\partial t}\underbrace{\vnabla \times \vE}_{\substack{-\frac{\partial \vB}{\partial t} \\ \text{Pers. Ke-3} \\\text{Maxwell}}}\\
        &=\vnabla \times \vJ + \epsilon_0\frac{\partial}{\partial t}\left(-\frac{\partial\vB}{\partial t}\right)\\
        &=\vnabla \times \vJ - \epsilon_0\frac{\partial^2\vB}{\partial t}
    \end{align*}
    karena $\vB=\mu_0 \vH$, maka
    \begin{align*}
        -\nabla^2\vH&=\vnabla \times \vJ - \epsilon_0\mu_0\frac{\partial^2\vH}{\partial t}
    \end{align*}
    Dari hukum Ohm $\vJ=\sigma \vE \neq 0$, untuk $\vE\neq 0$, sehingga persamaan diatas menjadi
    \begin{align*}
        -\nabla^2\vH&=\vnabla \times (\sigma\vE) - \epsilon_0\mu_0\frac{\partial^2\vH}{\partial t}\\
        &=\sigma \underbrace{\vnabla \times \vE}_{\substack{-\frac{\partial \vB}{\partial t} \\ \text{Pers. Ke-3} \\\text{Maxwell}}} - \epsilon_0\mu_0\frac{\partial^2\vH}{\partial t}\\
        -\nabla^2\vH&=-\sigma \frac{\partial \vB}{\partial t} - \epsilon_0\mu_0\frac{\partial^2\vH}{\partial t}
    \end{align*}
    karena $\vB=\mu_0 \vH$, maka
    \begin{align*}
        -\nabla^2\vH&=-\sigma\mu_0 \frac{\partial \vH}{\partial t} - \epsilon_0\mu_0\frac{\partial^2\vH}{\partial t}
    \end{align*}
    \textit{rearrange} persamaan diatas menjadi
    \begin{align*}
        \nabla^2\vH-\sigma\mu_0 \frac{\partial \vH}{\partial t} - \epsilon_0\mu_0\frac{\partial^2\vH}{\partial t}&=0\\
        \left(\nabla^2-\sigma\mu_0 \frac{\partial}{\partial t} - \epsilon_0\mu_0\frac{\partial^2}{\partial t}\right)\vH&=0\\
    \end{align*}
\end{itemize}
    % \item \textit{skin depth} didefinisikan sebagai jarak untuk mengurangi amplitudo Gelombang EM dengan faktor $1/e$, yakni:
\begin{equation*}
    \delta \equiv \frac{1}{\kappa};\ \text{dimana } \kappa \equiv \omega \sqrt{\frac{\epsilon\mu}{2}} \left[\sqrt{1+\left(\frac{\sigma}{\epsilon\omega}\right)^2}-1\right]^{1/2}
\end{equation*}
Untuk konduktivitas tinggi $(\sigma \gg \omega \epsilon) \Rightarrow \sigma/\epsilon \gg 1$, sehingga
\begin{align*}
    \delta &= \frac{1}{
        \omega \sqrt{\dfrac{\mu \epsilon}{2}} \left[\dfrac{\sigma}{\epsilon\omega}\right]^{1/2}
    }\\
    &=\frac{1}{\sqrt{\dfrac{\omega^{\cancel{2}} \mu \cancel{\epsilon}\sigma}{2\cancel{\epsilon}\cancel{\omega}}}}\\
    &=\frac{1}{\sqrt{\dfrac{\omega\mu\sigma}{2}}}\\
    &=\sqrt{\frac{2}{\omega\mu\sigma}}\\
    &= \sqrt{\frac{2}{2\pi f \mu \sigma}}
\end{align*}
diketahui nilai $f=10^{10}\si{Hz}$, $\sigma=3\times 10^7 \si{S/m}$ dan permeabilitas material $\mu=\mu_0(1+\chi_m)$, dimana suseptabilitas material perak $\chi_m=-2,4\times 10^{-5}$, atau
\begin{align*}
    \mu&=\mu_0(1+\chi_m)\\
    &=4\pi\times 10^{-7}(1+-2.4\times 10^{-5})\\
    &=1,26 \times 10^{-6}\si{N/A^2}
\end{align*}
subtitusi pada persamaan \textit{skin depth} sebelumnya, didapat
\begin{align*}
    \delta&= \sqrt{\frac{2}{2\pi (10^{10}) (1,26 \times 10^{-6}) (3\times 10^7)}}\\
    &= \sqrt{\frac{2}{2.37\times 10^{12}}}\\
    &=9,18 \times 10^{-7} \si{m}\\
    &=0,918\ \si{\mu m} 
\end{align*}
    % \item Untuk bahan dengan konduktifitas tinggi maka \textit{skin depth} nya adalah
\begin{align*}
    \delta &= \sqrt{\frac{2}{\omega \mu \sigma}}\\
           &= \sqrt{\frac{2}{2 \pi f \mu \sigma}}
\end{align*}
untuk mencari nilai frekuensi, maka berdasarkan persamaan diatas
\begin{align*}
    \sqrt{2\pi f \mu \sigma} &= \frac{\sqrt{2}}{\delta}\\
    \sqrt{f} &= \frac{\sqrt{2}}{\delta\sqrt{2\pi\mu \sigma}}\\
    f &= \frac{2}{\delta^2 2\pi\mu \sigma}\\
\end{align*}
diketahui bahwa konduktifitas $\sigma=3\times 10^7$, $\mu=\mu_0$, dan $\delta = 1\ \si{m}$, maka
\begin{align*}
    f&=\frac{2}{1^2 (2\pi) (4\pi\times 10^{-7}) (3\times 10^7)}\\
     &=0,00844 Hz\\
     &=8,44 \times 10^{-3}\ \si{Hz}
\end{align*}



\end{enumerate}
\end{document}

